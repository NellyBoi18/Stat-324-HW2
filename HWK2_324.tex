% Options for packages loaded elsewhere
\PassOptionsToPackage{unicode}{hyperref}
\PassOptionsToPackage{hyphens}{url}
%
\documentclass[
]{article}
\usepackage{amsmath,amssymb}
\usepackage{lmodern}
\usepackage{iftex}
\ifPDFTeX
  \usepackage[T1]{fontenc}
  \usepackage[utf8]{inputenc}
  \usepackage{textcomp} % provide euro and other symbols
\else % if luatex or xetex
  \usepackage{unicode-math}
  \defaultfontfeatures{Scale=MatchLowercase}
  \defaultfontfeatures[\rmfamily]{Ligatures=TeX,Scale=1}
\fi
% Use upquote if available, for straight quotes in verbatim environments
\IfFileExists{upquote.sty}{\usepackage{upquote}}{}
\IfFileExists{microtype.sty}{% use microtype if available
  \usepackage[]{microtype}
  \UseMicrotypeSet[protrusion]{basicmath} % disable protrusion for tt fonts
}{}
\makeatletter
\@ifundefined{KOMAClassName}{% if non-KOMA class
  \IfFileExists{parskip.sty}{%
    \usepackage{parskip}
  }{% else
    \setlength{\parindent}{0pt}
    \setlength{\parskip}{6pt plus 2pt minus 1pt}}
}{% if KOMA class
  \KOMAoptions{parskip=half}}
\makeatother
\usepackage{xcolor}
\usepackage[margin=1in]{geometry}
\usepackage{color}
\usepackage{fancyvrb}
\newcommand{\VerbBar}{|}
\newcommand{\VERB}{\Verb[commandchars=\\\{\}]}
\DefineVerbatimEnvironment{Highlighting}{Verbatim}{commandchars=\\\{\}}
% Add ',fontsize=\small' for more characters per line
\usepackage{framed}
\definecolor{shadecolor}{RGB}{248,248,248}
\newenvironment{Shaded}{\begin{snugshade}}{\end{snugshade}}
\newcommand{\AlertTok}[1]{\textcolor[rgb]{0.94,0.16,0.16}{#1}}
\newcommand{\AnnotationTok}[1]{\textcolor[rgb]{0.56,0.35,0.01}{\textbf{\textit{#1}}}}
\newcommand{\AttributeTok}[1]{\textcolor[rgb]{0.77,0.63,0.00}{#1}}
\newcommand{\BaseNTok}[1]{\textcolor[rgb]{0.00,0.00,0.81}{#1}}
\newcommand{\BuiltInTok}[1]{#1}
\newcommand{\CharTok}[1]{\textcolor[rgb]{0.31,0.60,0.02}{#1}}
\newcommand{\CommentTok}[1]{\textcolor[rgb]{0.56,0.35,0.01}{\textit{#1}}}
\newcommand{\CommentVarTok}[1]{\textcolor[rgb]{0.56,0.35,0.01}{\textbf{\textit{#1}}}}
\newcommand{\ConstantTok}[1]{\textcolor[rgb]{0.00,0.00,0.00}{#1}}
\newcommand{\ControlFlowTok}[1]{\textcolor[rgb]{0.13,0.29,0.53}{\textbf{#1}}}
\newcommand{\DataTypeTok}[1]{\textcolor[rgb]{0.13,0.29,0.53}{#1}}
\newcommand{\DecValTok}[1]{\textcolor[rgb]{0.00,0.00,0.81}{#1}}
\newcommand{\DocumentationTok}[1]{\textcolor[rgb]{0.56,0.35,0.01}{\textbf{\textit{#1}}}}
\newcommand{\ErrorTok}[1]{\textcolor[rgb]{0.64,0.00,0.00}{\textbf{#1}}}
\newcommand{\ExtensionTok}[1]{#1}
\newcommand{\FloatTok}[1]{\textcolor[rgb]{0.00,0.00,0.81}{#1}}
\newcommand{\FunctionTok}[1]{\textcolor[rgb]{0.00,0.00,0.00}{#1}}
\newcommand{\ImportTok}[1]{#1}
\newcommand{\InformationTok}[1]{\textcolor[rgb]{0.56,0.35,0.01}{\textbf{\textit{#1}}}}
\newcommand{\KeywordTok}[1]{\textcolor[rgb]{0.13,0.29,0.53}{\textbf{#1}}}
\newcommand{\NormalTok}[1]{#1}
\newcommand{\OperatorTok}[1]{\textcolor[rgb]{0.81,0.36,0.00}{\textbf{#1}}}
\newcommand{\OtherTok}[1]{\textcolor[rgb]{0.56,0.35,0.01}{#1}}
\newcommand{\PreprocessorTok}[1]{\textcolor[rgb]{0.56,0.35,0.01}{\textit{#1}}}
\newcommand{\RegionMarkerTok}[1]{#1}
\newcommand{\SpecialCharTok}[1]{\textcolor[rgb]{0.00,0.00,0.00}{#1}}
\newcommand{\SpecialStringTok}[1]{\textcolor[rgb]{0.31,0.60,0.02}{#1}}
\newcommand{\StringTok}[1]{\textcolor[rgb]{0.31,0.60,0.02}{#1}}
\newcommand{\VariableTok}[1]{\textcolor[rgb]{0.00,0.00,0.00}{#1}}
\newcommand{\VerbatimStringTok}[1]{\textcolor[rgb]{0.31,0.60,0.02}{#1}}
\newcommand{\WarningTok}[1]{\textcolor[rgb]{0.56,0.35,0.01}{\textbf{\textit{#1}}}}
\usepackage{longtable,booktabs,array}
\usepackage{calc} % for calculating minipage widths
% Correct order of tables after \paragraph or \subparagraph
\usepackage{etoolbox}
\makeatletter
\patchcmd\longtable{\par}{\if@noskipsec\mbox{}\fi\par}{}{}
\makeatother
% Allow footnotes in longtable head/foot
\IfFileExists{footnotehyper.sty}{\usepackage{footnotehyper}}{\usepackage{footnote}}
\makesavenoteenv{longtable}
\usepackage{graphicx}
\makeatletter
\def\maxwidth{\ifdim\Gin@nat@width>\linewidth\linewidth\else\Gin@nat@width\fi}
\def\maxheight{\ifdim\Gin@nat@height>\textheight\textheight\else\Gin@nat@height\fi}
\makeatother
% Scale images if necessary, so that they will not overflow the page
% margins by default, and it is still possible to overwrite the defaults
% using explicit options in \includegraphics[width, height, ...]{}
\setkeys{Gin}{width=\maxwidth,height=\maxheight,keepaspectratio}
% Set default figure placement to htbp
\makeatletter
\def\fps@figure{htbp}
\makeatother
\setlength{\emergencystretch}{3em} % prevent overfull lines
\providecommand{\tightlist}{%
  \setlength{\itemsep}{0pt}\setlength{\parskip}{0pt}}
\setcounter{secnumdepth}{-\maxdimen} % remove section numbering
\ifLuaTeX
  \usepackage{selnolig}  % disable illegal ligatures
\fi
\IfFileExists{bookmark.sty}{\usepackage{bookmark}}{\usepackage{hyperref}}
\IfFileExists{xurl.sty}{\usepackage{xurl}}{} % add URL line breaks if available
\urlstyle{same} % disable monospaced font for URLs
\hypersetup{
  pdftitle={Stat 324 Homework \#2 Due Wednesday February 8th 9 am},
  pdfauthor={Nelson Lu},
  hidelinks,
  pdfcreator={LaTeX via pandoc}}

\title{Stat 324 Homework \#2 Due Wednesday February 8th 9 am}
\author{Nelson Lu}
\date{}

\begin{document}
\maketitle

*Submit your homework to Canvas by the due date and time. Email your
lecturer if you have extenuating circumstances and need to request an
extension.

*If an exercise asks you to use R, include a copy of the code and
output. Please edit your code and output to be only the relevant
portions.

*If a problem does not specify how to compute the answer, you many use
any appropriate method. I may ask you to use R or use manually
calculations on your exams, so practice accordingly.

*You must include an explanation and/or intermediate calculations for an
exercise to be complete.

*Be sure to submit the HWK2 Autograde Quiz which will give you
\textasciitilde20 of your 40 accuracy points.

*50 points total: 40 points accuracy, and 10 points completion

\hypertarget{summarizing-data-numerically-and-graphically-i}{%
\subsection{Summarizing Data Numerically and Graphically
(I)}\label{summarizing-data-numerically-and-graphically-i}}

\textbf{Exercise 1:} A company that manufactures toilets claims that its
new pressure-assisted toilet reduces the average amount of water used by
more than 0.5 gallons per flush when compared to its current model. The
company selects 20 toilets of the \emph{current} type and 19 of the
\emph{new} type and measures the amount of water used when each toilet
is flushed once. The number of gallons measured for each flush are
recorded below. The measurements are also given in flush.csv.

Current Model: 1.63, 1.25, 1.23, 1.49, 2.11, 1.48, 1.94, 1.72, 1.85,
1.54, 1.67, 1.76, 1.46, 1.32, 1.23, 1.67, 1.74, 1.63, 1.25, 1.56

New Model: 1.28, 1.19, 0.90, 1.24, 1.00, 0.80, 0.71, 1.03, 1.27, 1.14,
1.36, 0.91, 1.09, 1.36, 0.91, 0.91, 0.86, 0.93, 1.36

\begin{quote}
\begin{enumerate}
\def\labelenumi{\alph{enumi}.}
\tightlist
\item
  Use R to create histograms to display the sample data from each model
  (any kind of histogram that you want since sample sizes are very
  similar). Have identical x and y axis scales so the two groups' values
  are more easily compared. Include useful titles.
\end{enumerate}
\end{quote}

\vspace{.5cm}

\begin{quote}
\begin{enumerate}
\def\labelenumi{\alph{enumi}.}
\setcounter{enumi}{1}
\tightlist
\item
  Compare the shape of the gallons per flush from the two types of
  toilets observed in this experiment.
\end{enumerate}
\end{quote}

\vspace{.5cm}

\begin{quote}
\begin{enumerate}
\def\labelenumi{\alph{enumi}.}
\setcounter{enumi}{2}
\tightlist
\item
  Compute the mean and median gallons flushed for the Current and New
  Model toilets using the built-in R functions. Compare the measures of
  center across the two groups and comment on how that relationship is
  evident in the histograms.
\end{enumerate}
\end{quote}

\vspace{.5cm}

\begin{quote}
\begin{enumerate}
\def\labelenumi{\alph{enumi}.}
\setcounter{enumi}{3}
\tightlist
\item
  Compute (using built-in R function) and compare the sample standard
  deviation of gallons flushed by the current and new model toilets.
  Comment on how the relative size of these values can be identified
  from the histograms.
\end{enumerate}
\end{quote}

\vspace{.5cm}

\begin{quote}
\begin{enumerate}
\def\labelenumi{\alph{enumi}.}
\setcounter{enumi}{4}
\tightlist
\item
  Use R to create side-by-side boxplots of the two sets in R so they are
  easily comparable.
\end{enumerate}
\end{quote}

\vspace{.5cm}

\begin{quote}
\begin{enumerate}
\def\labelenumi{\alph{enumi}.}
\setcounter{enumi}{5}
\tightlist
\item
  Explain why there are no values shown as a dot on the Current Model
  flush boxplot. To what values do the Current model flush boxplot
  whiskers extend? (Use R for your boxplot calculations and type=2 for
  quantiles)
\end{enumerate}
\end{quote}

\vspace{.5cm}

\begin{quote}
\begin{enumerate}
\def\labelenumi{\alph{enumi}.}
\setcounter{enumi}{6}
\tightlist
\item
  What would be the mean and median gallons flushed if we combined the
  two data sets into one large data set with 39 observations? Show how
  the mean can be calculated from all observations in one vector and
  also with the summary measures in part (c) along with the sample
  sizes. Explain why the median of the combined set cannot be computed
  based on (c).
\end{enumerate}
\end{quote}

\vspace{.5cm}

\textbf{Exercise 2} You are adding Badger-themed bedazzle to your
striped overalls and are using both red and white beads. You are
interested in how the size of the bag of beads you select your beads
from changes the probability of outcomes of interest. Compute the
probability for outcomes a and b using three different sampling
strategies each time.

(Small Pop) drawing without replacement from a small population where
the bag of beads contains 6 White beads and 4 Red beads.

(Large Pop) drawing without replacement from a large population where
the bag of beads contains 600 White beads and 400 Red beads.

(Same Pop) drawing from a population where the bag of beads always
contains 60\% White and 40\% Red beads.

Example: Consider choosing 3 beads. Calculate the probability of
selecting no white beads.

Small Pop:
P({[}RRR{]})=\(\frac{4}{10}*\frac{3}{9}*\frac{2}{8}=0.03333333\)

Large Pop:
P({[}RRR{]})=\(\frac{400}{1000}*\frac{399}{999}*\frac{398}{998}=0.06371181\)

Same Pop: P({[}RRR{]})=\(0.40*.40*.40=0.064\)

\begin{Shaded}
\begin{Highlighting}[]
\NormalTok{(}\DecValTok{4}\SpecialCharTok{/}\DecValTok{10}\NormalTok{)}\SpecialCharTok{*}\NormalTok{(}\DecValTok{3}\SpecialCharTok{/}\DecValTok{9}\NormalTok{)}\SpecialCharTok{*}\NormalTok{(}\DecValTok{2}\SpecialCharTok{/}\DecValTok{8}\NormalTok{)}
\end{Highlighting}
\end{Shaded}

\begin{verbatim}
## [1] 0.03333333
\end{verbatim}

\begin{Shaded}
\begin{Highlighting}[]
\NormalTok{(}\DecValTok{400}\SpecialCharTok{/}\DecValTok{1000}\NormalTok{)}\SpecialCharTok{*}\NormalTok{(}\DecValTok{399}\SpecialCharTok{/}\DecValTok{999}\NormalTok{)}\SpecialCharTok{*}\NormalTok{(}\DecValTok{398}\SpecialCharTok{/}\DecValTok{998}\NormalTok{)}
\end{Highlighting}
\end{Shaded}

\begin{verbatim}
## [1] 0.06371181
\end{verbatim}

\begin{Shaded}
\begin{Highlighting}[]
\NormalTok{.}\DecValTok{4}\SpecialCharTok{*}\NormalTok{.}\DecValTok{4}\SpecialCharTok{*}\NormalTok{.}\DecValTok{4}
\end{Highlighting}
\end{Shaded}

\begin{verbatim}
## [1] 0.064
\end{verbatim}

\begin{quote}
\begin{enumerate}
\def\labelenumi{\alph{enumi}.}
\tightlist
\item
  Consider choosing 3 beads. Calculate the probability of selecting
  exactly 1 white bead.
\end{enumerate}
\end{quote}

\textbf{Small Pop: }

\textbf{Large Pop: }

\textbf{Same Pop: }

\begin{quote}
\begin{enumerate}
\def\labelenumi{\alph{enumi}.}
\setcounter{enumi}{1}
\tightlist
\item
  Consider choosing 3 beads. Calculate the probability of selecting at
  least 1 white bead.
\end{enumerate}
\end{quote}

\textbf{Small Pop: }

\textbf{Large Pop: }

\textbf{Same Pop: }

\begin{quote}
\begin{enumerate}
\def\labelenumi{\alph{enumi}.}
\setcounter{enumi}{2}
\tightlist
\item
  Consider sampling without replacement. Does drawing from a population
  that is \textbf{small} or \textbf{large} relative to your sample size
  result in an probability that is closest to the probability when
  sampling with replacement?
\end{enumerate}
\end{quote}

\vspace{.5cm}

\textbf{Exercise 3} Six hundred (600) paving stones were examined for
cracking and discoloration. Eighteen (18) were found to be cracked and
24 were found to be discolored. A total of 562 stones were neither
cracked nor discolored.

\begin{quote}
\begin{enumerate}
\def\labelenumi{\alph{enumi}.}
\tightlist
\item
  Create a 2-way table to organize the counts of stones in each of the 4
  combinations of Cracked/Not Cracked and Discolored/Not Discolored.
\end{enumerate}
\end{quote}

\begin{longtable}[]{@{}cccc@{}}
\toprule()
X & Cracked & Not Cracked & Total \\
\midrule()
\endhead
Discolored & & & 24 \\
Not Discolored & & 562 & \\
Total & 18 & & 600 \\
\bottomrule()
\end{longtable}

\begin{quote}
\begin{enumerate}
\def\labelenumi{\alph{enumi}.}
\setcounter{enumi}{1}
\tightlist
\item
  What is the probability that a randomly sampled paving stone from this
  set is discolored and not cracked?
\end{enumerate}
\end{quote}

\vspace{.5cm}

\begin{quote}
\begin{enumerate}
\def\labelenumi{\alph{enumi}.}
\setcounter{enumi}{2}
\tightlist
\item
  What is the probability that a randomly sampled paving stone from this
  group is cracked or discolored?
\end{enumerate}
\end{quote}

\vspace{.5cm}

\begin{quote}
\begin{enumerate}
\def\labelenumi{\alph{enumi}.}
\setcounter{enumi}{3}
\tightlist
\item
  What is the probability that in a random sample of 3 paving stones
  from this set without replacement, at least one of the three is
  cracked or discolored?
\end{enumerate}
\end{quote}

\vspace{.5cm}

\begin{quote}
(OPTIONAL) Explain what the code in this simulation below is doing and
how it can be used to check your answer from (d).
\end{quote}

\begin{Shaded}
\begin{Highlighting}[]
\NormalTok{pop}\OtherTok{=}\FunctionTok{c}\NormalTok{(}\FunctionTok{rep}\NormalTok{(}\DecValTok{1}\NormalTok{,}\DecValTok{38}\NormalTok{), }\FunctionTok{rep}\NormalTok{(}\DecValTok{0}\NormalTok{,}\DecValTok{562}\NormalTok{))}
\NormalTok{nsamp}\OtherTok{=}\DecValTok{500000}
\NormalTok{num\_CorD}\OtherTok{=}\FunctionTok{rep}\NormalTok{(}\DecValTok{9}\NormalTok{,nsamp)}
\ControlFlowTok{for}\NormalTok{ (i }\ControlFlowTok{in} \DecValTok{1}\SpecialCharTok{:}\NormalTok{nsamp)\{}
\NormalTok{  samp}\OtherTok{=}\FunctionTok{sample}\NormalTok{(pop,}\DecValTok{3}\NormalTok{, }\AttributeTok{replace=}\ConstantTok{FALSE}\NormalTok{)}
\NormalTok{  num\_CorD[i]}\OtherTok{=}\FunctionTok{sum}\NormalTok{(samp)}
\NormalTok{\}}
\FunctionTok{hist}\NormalTok{(num\_CorD, }\AttributeTok{labels=}\ConstantTok{TRUE}\NormalTok{)}

\FunctionTok{sum}\NormalTok{(num\_CorD}\SpecialCharTok{==}\DecValTok{1}\SpecialCharTok{|}\NormalTok{num\_CorD}\SpecialCharTok{==}\DecValTok{2}\SpecialCharTok{|}\NormalTok{num\_CorD}\SpecialCharTok{==}\DecValTok{3}\NormalTok{)}\SpecialCharTok{/}\NormalTok{nsamp}
\FunctionTok{sum}\NormalTok{(num\_CorD}\SpecialCharTok{\textgreater{}=}\DecValTok{1}\NormalTok{)}\SpecialCharTok{/}\NormalTok{nsamp}
\DecValTok{1}\SpecialCharTok{{-}}\FunctionTok{sum}\NormalTok{(num\_CorD}\SpecialCharTok{==}\DecValTok{0}\NormalTok{)}\SpecialCharTok{/}\NormalTok{nsamp}
\end{Highlighting}
\end{Shaded}

\vspace{.5cm}

\begin{quote}
\begin{enumerate}
\def\labelenumi{\alph{enumi}.}
\setcounter{enumi}{4}
\tightlist
\item
  What is the probability that a randomly sampled paving stone from this
  group has discoloration, given we know that it is cracked?
\end{enumerate}
\end{quote}

\vspace{.5cm}

\begin{quote}
\begin{enumerate}
\def\labelenumi{\alph{enumi}.}
\setcounter{enumi}{5}
\tightlist
\item
  Is being discolored and cracked independent in this set of 600 paving
  stones?
\end{enumerate}
\end{quote}

\vspace{.5cm}

\begin{quote}
\begin{enumerate}
\def\labelenumi{\alph{enumi}.}
\setcounter{enumi}{6}
\tightlist
\item
  Now suppose in another group of 600 paving stones, forty-eight (48)
  were found to be cracked and 25 were found to be discolored. How many
  stones would be cracked and discolored (??) if the events: discolored
  and cracked are independent in this group of 600 stones? Make sure to
  show how you calculated your answer.
\end{enumerate}
\end{quote}

\begin{longtable}[]{@{}cccc@{}}
\toprule()
X & Cracked & Not Cracked & Total \\
\midrule()
\endhead
Discolored & ?? & & 25 \\
Not Discolored & & & \\
Total & 48 & 552 & 600 \\
\bottomrule()
\end{longtable}

\end{document}
